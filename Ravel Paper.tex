\documentclass[english,12pt,letterpaper]{article}
\usepackage[T1]{fontenc}
\usepackage[left=1in,right=1in,top=1in,bottom=1in]{geometry}
\usepackage{babel}
\usepackage{fancyhdr}
\usepackage{csquotes}
\usepackage[doublespacing]{setspace}
\usepackage{musicography}
\usepackage[colorlinks=true,allcolors=blue]{hyperref}

\usepackage[notes,backend=biber]{biblatex-chicago}
\addbibresource{Bibliography.bib}

\title{\textbf{Ravel Paper}}
\author{Tif{}fany Lin}
\date{December 12, 2025}

\begin{document}
\maketitle

% \begin{enumerate}
%   \item Ravel's life
%   \begin{itemize}
%     \item Early years
%     \item Influences from others (Debussy, Fauré)
%     \item Prix de Rome
%   \end{itemize}
%   \item Ravel's compositions and musical style
%   \begin{itemize}
%     \item Characteristics of his style
%   \end{itemize}
%   \item Ravel's piano works
%   \item Pavane piano version analysis
%   \begin{itemize}
%     \item Title story
%     \item Pavane dance form
%     \item Structure and form
%   \end{itemize}

\newcommand{\Pavane}{\textit{Pavane}}

  Maurce Ravel's \textit{Pavane pour une infante défunte} (composed in 1899) was one of his first published pieces (1900).
  The piece was commissioned and dedicated to Winnaretta Singer, Princesse Edmond de Polignac, a major patron of the arts in Paris.
  \Autocite{Pavane}
  It was officially premiered in 1902 by the pianist Ricardo Viñes, and was well received by the audience.
  \Autocite{Pavane}
  Ravel later explained that he chose the title due to its alliteration, in that the music evokes `a pavan that a little princess might, in former times, have danced at the Spanish court.'
  \Autocite{Pavane}

  The works of Edgar Allan Poe had a significant influence on Ravel due to its compositional methods and aesthetic ideals.
  \Autocite{Larner}
  He later declared that `my greatest teacher in composition was Edgar Allan Poe'.
  \Autocite[\textit{Maurice Ravel}, 41]{Larner}
  However, in relation to the \Pavane, Poe argued that melancholy is the most legitimate poetic tone and that the death of a beautiful woman is the most poetic subject.
  \Autocite[\textit{Maurice Ravel}, 41]{Larner}
  Ravel did title it \Pavane{} in accordance with Poe's topic, but it was only chosen  because he liked the sound of it.
  \Autocite[\textit{Maurice Ravel}, 59]{Larner}
  Ravel ultimately found a deeper source of true melancholy later in his career, in which more geared towards human guilt and suffering shown by the poet, Paul Verlaine's imprisonment.
  \Autocite[\textit{Maurice Ravel}, 41]{Larner}

  With the \Pavane's success, Ravel orchestrated it for a small orchestra in 1910, premiering in 1911, \Autocite{Pavane} which became even more popular after its orchestration.
  \Autocite[\textit{Maurice Ravel}, 60]{Larner}
  Despite the \Pavane{} becoming popular in salons and among amateur pianists, by 1912,\Autocite{Pavane} Ravel has his own reflections on it that is quite harsh and self-critical.
  \Autocite{Roland-Manuel}
  He criticized the piece for showing too much imitation of Chabrier and Fauré's style,\Autocite[\textit{Maurice Ravel}, 60]{Larner} and not much originality.
  \Autocite[\textit{Maurice Ravel}, 28--29]{Roland-Manuel}
  Along with the critiques, the \Pavane's fame was attributed less to the composition itself and towards the interpretations of its performers.
  \Autocite[\textit{Maurice Ravel}, 28--29]{Roland-Manuel}

  The \Pavane{} reflects a 16th century slow processional dance.
  \Autocite{Pavane}
  The use of classical dance forms was part of a broader revival among other French composers such as Saint-Saëns.
  \Autocite{Pavane}
  Although Ravel emphasized that his primary influence was Chabrier, his admiration for Debussy and influence is also apparent.
  \Autocite{Mawer}
  For example, the parallel ninths in the 1899 \Pavane{} is quite similar to that found in Debussy's 1896 `Sarabande,' which Ravel later orchestrated \Autocite[\textit{Maurice Ravel}, 228]{Larner} in 1922.
  \Autocite[\textit{The Cambridge Companion to Ravel}, 72--73]{Mawer}
  In regards to Chabrier, his 1887 opera, \textit{Le Roi malgré lui} include successions of both diatonic and chromatic ninth chords.
  This technique that Chabrier uses can be stemmed from Chopin's \textit{Nouvelle étude} in D\musFlat, which uses sequential sevenths.
  \Autocite[\textit{The Cambridge Companion to Ravel}, 72--73]{Mawer}
  Howat writes that Ravel, Debussy, and Chabrier all use a technique called harmonic ellipsis. This means they ignore the typical classical resolution of a chord, instead immediately jumping to the next ninth chord.
  \Autocite[\textit{The Cambridge Companion to Ravel}, 72--73]{Mawer}
  Ultimately, this technique of harmonic ellipses became a defining element of Ravel's own composition style.
  \Autocite[\textit{The Cambridge Companion to Ravel}, 72--73]{Mawer}

  The issue of the tempo in the \Pavane{} has been a subject of debate in regards to the performance and interpretation of it.
  Piano editions prior to 1913 have the metronome marking as \musQuarter{} = 80, while the tempo of editions after 1913 have been reduced to \musQuarter{} = 54, which is mirrored in the 1912 orchestral score.
  \Autocite{Pavane}
  The absence of corrections in Ravel's personal copy and the lack of tempo indications in the orchestral autograph complicate this matter.
  \Autocite{Pavane}
  Ravel's sarcastic comment to Charles Oulmont who performed the \Pavane{}, saying that Oulmont ``wrote a `Pavane for a Dead Princess' not a `Dead Pavane for a Princess,''' \Autocite{Pavane} suggests that Ravel resisted excessively slow interpretations.
  To further complicate this matter, Ravel's own piano roll recording in 1922, when analyzed, fluctuates by more than 10bpm, and the dynamics diverges from the printed music.
  \Autocite{Pavane}
  This suggests that both the tempo and dynamics in \Pavane{} should be flexible and up to the interpretation of the performer.

  % \item Ravel's orchestration
  % \begin{itemize}
  %   \item Orchestrations Ravel had done
  %   \item Techniques
  % \end{itemize}
  % \item Pavane orchestral version analysis

  % \begin{itemize}
  %   \item Differences between piano and orchestral versions
  %   \item Instrumental choices
  % \end{itemize}
  % \item Legacy and reception
  % \begin{itemize}
  %   \item Pavane in popular media
  % \end{itemize}

  \textbf{Your Lie in April}

  \Pavane appears in \textit{Your Lie in April} as an important and symbolic representation in the series, representing an important theme(?).
  Already in the piece's title, \textit{Pavane pour une infante défunte}, the meaning behind is already shown.
  [Despite not literally about dead princess]
  In which ``pavane'' is a \href{https://en.wikipedia.org/wiki/Pavane}{slow dance} and the ``princess'' referring to Kaori.
  However, Kaori's health deteriorates, leading to the ``dead princess'' referred to in the title.

  In episode 16, Pavane starts to play at 20:33 and at 21:15, Kaori asks ``Want to commit double suicide?'' quoted from Masahiro Mita's ``Ichigo Doumei.''
  
  [``T/N: A Japanese novel published in 1990 about a suicidal boy who meets a girl in the hospital.
  Kaori is quoting from it.''
  \href{https://bato.to/chapter/835932}{bato ch32, Easy Going Scans}]

  [\href{https://www.reddit.com/r/ShigatsuwaKiminoUso/comments/7076cj/spoiler_after_reading_ichigo_doumei/}{ichigo doumei context}]


  In episode 17, 6:58, which correlates to chapter 33, Kousei hears Pavane being played on the way home and runs away in denial, saying ``I don't want to hear it... I don't want to hear any stupid Ravel... I won't want to think about anything... I wish I could just stop hearing everything.''
  \href{https://bato.to/chapter/1559720}{bato ch33}

  [``T/N: Ravel was a French composer whose piece, ``Pavane for a Dead Princess,'' is heavily referenced in the Ichigo Alliance novel Kaori quotes from in Chapter 32.
  In the novel, the main character plays the song on piano for female lead, who is hospital-ridden and has had one of her legs amputated.''
  \href{https://bato.to/chapter/1559720}{bato ch33, Easy Going Scans}]

  Kousei isn't literally rejecting the composer, he is rejecting the fate the piece represents.
  So in episode 18, 18:52, Kousei responds with ``I can't commit double suicide with you.'
  He is refusing to accept that Kaori will be the "dead princess" and he will not be playing Ravel for her `funeral.'
  \href{https://www.reddit.com/r/YourLieinApril/comments/n55qhu/no_way_am_i_ever_gonna_play_ravel/}{reddit}

  Furthermore, Kousei performs Rachmaninoff's arrangement of Tchaikovsky's ``Rose Adagio'' and ``Garland Waltz'' from ``The Sleeping Beauty'' as a stark contrast from the ``Pavane.''
  \href{https://bato.to/chapter/1559723}{ch36}

  \textbf{White Album 2}

  Early in the \textit{Introductory Chapter}, Haruki plays the melody of \Pavane while Kazusa plays the piano arrangement.
  This is the first time it's directly mentioned that the pianist accompanies Haruki.

  [``Haruki: The piano accompanied me many times, until I finally got into shape.
  And when it was pleased with my form, it accompanied my music.
  It was a mysterious sound that would pull pranks on me, guide me, and even show me my weaknesses simply through hearing it.''
  \href{https://todokanaitl.github.io/}{Todokanai Translations}]

% \end{enumerate}

\end{document}