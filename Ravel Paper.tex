\documentclass[english,12pt,letterpaper]{article}
\usepackage[T1]{fontenc}
\usepackage[left=1in,right=1in,top=1in,bottom=1in]{geometry}
\usepackage{babel}
\usepackage{fancyhdr}
\usepackage{csquotes}
\usepackage[doublespacing]{setspace}
\usepackage{musicography}
\usepackage[colorlinks=true,allcolors=blue]{hyperref}

\usepackage[notes,backend=biber]{biblatex-chicago}
\addbibresource{Bibliography.bib}

\title{\textbf{Ravel Paper}}
\author{Tif{}fany Lin}
\date{December 12, 2025}

\begin{document}
% \maketitle

% \begin{enumerate}
%   \item Ravel's life
%   \begin{itemize}
%     \item Early years
%     \item Influences from others (Debussy, Fauré)
%     \item Prix de Rome
%   \end{itemize}
%   \item Ravel's compositions and musical style
%   \begin{itemize}
%     \item Characteristics of his style
%   \end{itemize}
%   \item Ravel's piano works
%   \item Pavane piano version analysis
%   \begin{itemize}
%     \item Title story
%     \item Pavane dance form
%     \item Structure and form
%   \end{itemize}

\newcommand{\Pavane}{\textit{Pavane}}
\newcommand{\YLIA}{\textit{Your Lie in April}}
\newcommand{\WA}{\textit{White Album 2}}

  Maurice Ravel's \textit{Pavane pour une infante défunte} (composed in 1899) was one of his first published pieces (1900).
  The piece was commissioned and dedicated to Winnaretta Singer, Princesse Edmond de Polignac, a major patron of the arts in Paris.
  \Autocite{Pavane}
  It was officially premiered in 1902 by the pianist Ricardo Viñes, and was well received by the audience.
  \Autocite{Pavane}
  Ravel later explained that he chose the title due to its alliteration, in that the music evokes `a pavan that a little princess might, in former times, have danced at the Spanish court.'
  \Autocite{Pavane}
  This emphasis on sonority over the symbolic meaning reveals Ravel's early fascination with the musicality of language itself, a tendency that foreshadows his later mature style towards timbre and texture.

  The works of Edgar Allan Poe had a significant influence on Ravel due to its compositional methods and aesthetic ideals.
  \Autocite{Larner}
  He later declared that `my greatest teacher in composition was Edgar Allan Poe'.
  \Autocite[\textit{Maurice Ravel}, 43]{Larner}
  However, in relation to the \Pavane, Poe argued that melancholy is the most legitimate poetic tone and that the death of a beautiful woman is the most poetic subject.
  \Autocite[\textit{Maurice Ravel}, 43]{Larner}
  Ravel did title it \Pavane{} in accordance with Poe's topic, but it was only chosen  because he liked the sound of it.
  \Autocite[\textit{Maurice Ravel}, 59]{Larner}
  This tension between Poe's literary melancholy and Ravel's insistence that he liked the title undermines how the \Pavane{} resists being overly melancholy and instead a mean of portraying other emotions.
  Ravel ultimately found a deeper source of true melancholy later in his career, in which was shaped by themes of human guilt and suffering shown by the poet, Paul Verlaine's imprisonment.
  \Autocite[\textit{Maurice Ravel}, 43]{Larner}

  Despite Ravel's ambivalence towards an emotional interpretation, the work nonetheless achieved widespread popularity, prompting him to revisit it later on in his life.
  With the \Pavane's success, Ravel orchestrated it for a small orchestra in 1910, premiering in 1911, \Autocite{Pavane} which became even more popular after its orchestration.
  \Autocite[\textit{Maurice Ravel}, 60]{Larner}
  By giving the main melody to the horn, it creates a sense of nostalgia and somber timbre to intensify the melancholy.
  Despite the \Pavane{} becoming popular in salons and among amateur pianists, by 1912,\Autocite{Pavane} Ravel reflected on it with harsh self-criticism.
  \Autocite{Roland-Manuel}
  He criticized the piece for showing too much imitation of Chabrier and Fauré's style,\Autocite[\textit{Maurice Ravel}, 60]{Larner} and not much originality.
  \Autocite[\textit{Maurice Ravel}, 28--29]{Roland-Manuel}
  Along with the critiques, the \Pavane's fame was attributed less to the composition itself and to the interpretations of its performers.
  \Autocite[\textit{Maurice Ravel}, 28--29]{Roland-Manuel}

  To understand Ravel's criticism, it is necessary to place the piece within its historical and stylistic traditions.
  The \Pavane{} reflects a 16th century slow processional dance, \Autocite{Pavane} originating in Europe.
  It is said that the name comes from the Latin \textit{pavo} or French \textit{paon} meaning peacock.
  \Autocite{Goss}
  The dance was a slow processional dance, often performed at the weddings of girls of high status.
  \Autocite{Goss}
  The use of classical dance forms was part of a broader revival among other French composers such as Saint-Saëns.
  \Autocite{Pavane}
  From a stylistic standpoint, the \Pavane{} draws heavily on classical and 19th century precedents. The elements include a five-part rondo form, firmly rooted in G major with a brief turn to G minor, and reliance on classical compositional means (repetition, sequence, pedal point, contrary motion), and the frequent use of 9th chords.
  \Autocite{Kaminsky}
  The juxtaposition of traditional techniques the modern harmony illustrates how Ravel simultaneously honors classical structures while undermining them with modern sororities.
  Despite the convention framework, Ravel portrays the melodic development in a more mature style.
  The opening melody in measures 1--2 with the accompanying detached impression of a lute accompanying \Autocite{Demuth} leads to a continuation in measures 3--4, ending on a half cadence in measures 6--7.
  \Autocite[\textit{Unmasking Ravel}, 86--90]{Kaminsky}
  At measure 13, the first episode rises over a repeated chords over a pedal point.
  \Autocite{Demuth}
  At the cadence of that section, starting at measures 25, Ravel uses independently moving block dissonances of 9th and 13ths, a features of Debussy's works.
  \Autocite[\textit{Ravel}, 50--53]{Demuth}
  The opening theme returns again at measure 28, this time more complex, with large open voicings doubling the octave and 15th,
  and leads into the G minor section at measure 40, with a dominant 13th \Autocite[\textit{Ravel}, 50--53]{Demuth} at measure 42.
  The theme then returns for the final time at measure 60, this time in its most florid form with the lute effect doubled in 16ths.
  \Autocite[\textit{Ravel}, 50--53]{Demuth}
  In addition, in the final cadence section \textbf{En élargissant beaucoup} spanning measures 70--72, the pianist is faced with large voicings of chords, requiring careful pedal control.
  \Autocite[\textit{Ravel}, 50--53]{Demuth}
  The increasing complexity of the returning theme suggests the dance becoming more and ore ornate with each iteration, concluding to a final climatic scene.
  The \Pavane{} shows Ravel's technical skill in melodic development which foreshadows the \Autocite[\textit{Ravel}, 50--53]{Demuth} of his maturity in his later works.

  Although Ravel emphasized that his primary influence was Chabrier, his admiration for Debussy and influence is also apparent.
  \Autocite{Mawer}
  For example, the parallel ninths in the 1899 \Pavane{} is quite similar to that found in Debussy's 1896 `Sarabande,' which Ravel later orchestrated \Autocite[\textit{Maurice Ravel}, 228]{Larner} in 1922.
  \Autocite[\textit{The Cambridge Companion to Ravel}, 72--73]{Mawer}
  In regards to Chabrier, his 1887 opera, \textit{Le Roi malgré lui} include successions of both diatonic and chromatic ninth chords.
  This technique that Chabrier uses can be stemmed from Chopin's \textit{Nouvelle étude} in D\musFlat, which uses sequential sevenths.
  \Autocite[\textit{The Cambridge Companion to Ravel}, 72--73]{Mawer}
  Howat writes that Ravel, Debussy, and Chabrier all use a technique called harmonic ellipsis. This means they ignore the typical classical resolution of a chord, instead immediately jumping to the next ninth chord.
  \Autocite[\textit{The Cambridge Companion to Ravel}, 72--73]{Mawer}
  Ultimately, this technique of harmonic ellipses became a defining element of Ravel's own composition style.
  \Autocite[\textit{The Cambridge Companion to Ravel}, 72--73]{Mawer}
  Ravel's use of harmonic ellipses not only used under the influence of Debussy, but also shows a shift in French musical traditions towards a modern color and texture compared to the traditional functional harmonic progression.

  Interpretation of the \Pavane{} is also subject to the high ambiguity of the performance tempo.
  The issue of the tempo in the \Pavane{} has been a subject of debate in regards to the performance and interpretation of it.
  Piano editions prior to 1913 have the metronome marking as \musQuarter{} = 80, while the tempo of editions after 1913 have been reduced to \musQuarter{} = 54, which is mirrored in the 1912 orchestral score.
  \Autocite{Pavane}
  The absence of corrections in Ravel's personal copy and the lack of tempo indications in the orchestral autograph complicate this matter.
  \Autocite{Pavane}
  Ravel's sarcastic comment to Charles Oulmont who performed the \Pavane{}, saying that Oulmont ``wrote a `Pavane for a Dead Princess' not a `Dead Pavane for a Princess,''' \Autocite{Pavane} suggests that Ravel resisted excessively slow interpretations.
  To further complicate this matter, Ravel's own piano roll recording in 1922, when analyzed, fluctuates by more than 10bpm, and the dynamics diverges from the printed music \Autocite{Pavane} despite insisting that it should be played calmly without any \textit{rubato}.
  \Autocite[\textit{Ravel}, 50--53]{Demuth}
  This suggests that both the tempo and dynamics in \Pavane{} should be flexible and up to the interpretation of the performer.
  The conflicting tempo implies that Ravel valued expressive nuance in performance over strict precision, revealing an interpretive range for performers to imagine.

  % \item Ravel's orchestration
  % \begin{itemize}
  %   \item Orchestrations Ravel had done
  %   \item Techniques
  % \end{itemize}
  % \item Pavane orchestral version analysis

  % \begin{itemize}
  %   \item Differences between piano and orchestral versions
  %   \item Instrumental choices
  % \end{itemize}
  % \item Legacy and reception
  % \begin{itemize}
  %   \item Pavane in popular media
  % \end{itemize}

  % \textbf{Your Lie in April}

  The \Pavane{} is a key piece of music that appears in various forms of popular culture, most notably in the anime/manga series \YLIA{} and the visual novel \WA{}.
  In both works, the music is strategically placed not just as background sound, but it serves as a thematic and narrative device that addresses the characters' internal conflicts, growth, and relationships.
  This reflects how Ravel's composition, originally in the form of a dance, acquires new symbolic meaning when exposed to the narrative media, a medium in which emotional storytelling is achieved.

  In \YLIA{}, the \Pavane{} is used as a heavy symbol for the inevitable death of the female lead, Kaori Miyazono.
  The piece's title, despite Ravel's assertion that it is not literally about a dead princess, is used in such a way in the series to represent Kaori as the `dead princess' as the piece's title is directly linked to Kaori's deteriorating health.
  This reinterpretation demonstrates how cultural context can override the original intent, transforming Ravel's iconic title into a literal metaphor involving death.
  The piece is also used in conjunction with the Japanese novel \textit{Ichigo Doumei} by Masahiro Mita, a story about a suicidal boy who meets a hospitalized girl.
  \Autocite{IchiDou}
  In Episode 16, \Pavane{} starts to play at 20:33 and at 21:15, Kaori directly quotes from \textit{Ichigo Doumei} ``Want to commit double suicide?''
  The musical context in which \Pavane{} plays is in conjunction to her fate with the novel's tragic theme.
  
  % [``T/N: A Japanese novel published in 1990 about a suicidal boy who meets a girl in the hospital.
  % Kaori is quoting from it.''
  % \href{https://bato.to/chapter/835932}{bato ch32, Easy Going Scans}]

  % [\href{https://www.reddit.com/r/ShigatsuwaKiminoUso/comments/7076cj/spoiler_after_reading_ichigo_doumei/}{ichigo doumei context}]

  In Episode 17 (6:58), Kousei heard \Pavane{} being played on the way home and reacts with intense denial, running away and saying ````I don't want to hear it... I don't want to hear any stupid Ravel...''
  Kousei is not literally rejecting the composer, he is rejecting the tragedy that the piece entails, the fate the Kaori will be the `dead princess'.
  In Episode 18 (18:52), Kousei solidifies this rejection and responds with ``I can't commit double suicide with you.''
  This is his refusal to accept her fate and his refusal to be the one to play \Pavane{} for her funeral.
  Here, the music functions as an external force that the protagonist resists, undermining the tension between denial and acceptance.

  % which correlates to chapter 33, Kousei hears Pavane being played on the way home and runs away in denial, saying ``I don't want to hear it... I don't want to hear any stupid Ravel... I won't want to think about anything... I wish I could just stop hearing everything.''
  % \href{https://bato.to/chapter/1559720}{bato ch33}

  % [``T/N: Ravel was a French composer whose piece, ``Pavane for a Dead Princess,'' is heavily referenced in the Ichigo Alliance novel Kaori quotes from in Chapter 32.
  % In the novel, the main character plays the song on piano for female lead, who is hospital-ridden and has had one of her legs amputated.''
  % \href{https://bato.to/chapter/1559720}{bato ch33, Easy Going Scans}]

  % Kousei isn't literally rejecting the composer, he is rejecting the fate the piece represents.
  % So in episode 18, 18:52, Kousei responds with ``I can't commit double suicide with you.'
  % He is refusing to accept that Kaori will be the "dead princess" and he will not be playing Ravel for her `funeral.'
  % \href{https://www.reddit.com/r/YourLieinApril/comments/n55qhu/no_way_am_i_ever_gonna_play_ravel/}{reddit}

  Ultimately, Kousei performs Rachmaninoff's arrangement of Tchaikovsky's ``Rose Adagio'' and ``Garland Waltz'' from \textit{The Sleeping Beauty} as a stark contrast from the \Pavane{}.
  \textit{The Sleeping Beauty} is a fairy tale about a princess who wakes up from a long sleep, which symbolizes Kousei's wish for Kaori, to have a happy ending.
  This juxtaposition highlights how musical choices can rewrite narrative meaning in which Ravel's \Pavane{} highlights inevitability, while Tchaikovsky's \textit{The Sleeping Beauty} offers hope.
  % \href{https://bato.to/chapter/1559723}{ch36}

  % \textbf{White Album 2}

  In \WA{}, the \Pavane{} serves a different function.
  It is used to mark the beginning nature of the music and eventually troubled romantic relationship between Kitahara Haruki and Touma Kazusa.
  Early in the \textit{Introductory Chapter}, Haruki plays the melody of \Pavane while Touma plays the piano arrangement.
  This moment is significant because it is the first time it is directly mentioned that Touma accompanied Haruki.
  The music literally brought them together.
  The accompanying piano is described by Haruki as ``a mysterious sound that would pull pranks on me, guide me, and even show me my weaknesses simply through hearing it.''
  By using the \Pavane{}, a piece culturally associated with melancholy and a final fate, the narrative foreshadows the emotional tragic nature of their relationship.
  In this context the \Pavane{} shifts from symbolizing death to symbolizing intimacy, showing it's ability to adapt culturally.
  \Autocite{WA2}

  % [``Haruki: The piano accompanied me many times, until I finally got into shape.
  % And when it was pleased with my form, it accompanied my music.
  % It was a mysterious sound that would pull pranks on me, guide me, and even show me my weaknesses simply through hearing it.''
  % \href{https://todokanaitl.github.io/}{Todokanai Translations}]

  Ravel's \Pavane{} has outsourced its origins to become a cultural symbol in Japanese media.
  In \YLIA{}, it portrays death and denial, while in \WA{}, it represents an ultimately fated relationship.
  These two reinterpretations show the capacity of music to show different themes that are devoid of the original concept.
  These reinterpretations mirrors the \Pavane{}'s own historical ambiguity, from its misleading title to its performative tempo, it is suitable for interpretation across many genres.

  The cultural impact of Ravel's \Pavane{} even extends even to the realm of music engraving.
  Kazuhiro Hoshide's studio has developed a custom music notation font ``Pavane'', designed to create beautifully engraved scores.
  \Autocite{Hoshide}
  The font has been inspired by the elegance of  the peacock \Autocite{Hoshide} and as mentioned earlier by Goss's observation, the term ``pavane'' may have been historically used to describe a peacock.
  \Autocite{Goss}
  Coincidentally, two of the engraving samples featured on the website are Ravel's \Pavane{} in 6mm and 7mm stave sizes respectively.
  The engraving font \Pavane{} demonstrates how Ravel's composition continues to inspire continues to inspire cultural production beyond the pure musical context.

% \end{enumerate}

\end{document}