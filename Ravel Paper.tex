\documentclass[english,12pt,letterpaper]{article}
\usepackage[T1]{fontenc}
\usepackage[left=1in,right=1in,top=1in,bottom=1in]{geometry}
\usepackage{babel}
\usepackage{fancyhdr}
\usepackage{csquotes}
\usepackage[doublespacing]{setspace}
\usepackage{hyperref}

\title{\textbf{Ravel Paper}}
\author{Tif{}fany Lin}
\date{November 30, 2025}

\begin{document}
\maketitle

\begin{enumerate}
  \item Ravel's life
  \begin{itemize}
    \item Early years
    \item Influences from others (Debussy, Fauré)
    \item Prix de Rome
  \end{itemize}
  \item Ravel's compositions and musical style
  \begin{itemize}
    \item Characteristics of his style
  \end{itemize}
  \item Ravel's piano works
  \item Pavane piano version analysis
  \begin{itemize}
    \item Title story
    \item Pavane dance form
    \item Structure and form
  \end{itemize}
  \item Ravel's orchestration
  \begin{itemize}
    \item Orchestrations Ravel had done
    \item Techniques
  \end{itemize}
  \item Pavane orchestral version analysis
  \begin{itemize}
    \item Differences between piano and orchestral versions
    \item Instrumental choices
  \end{itemize}
  \item Legacy and reception
  \begin{itemize}
    \item Pavane in popular media
  \end{itemize}

  \textbf{Your Lie in April}

  \textit{Pavane} appears in \textit{Your Lie in April} as an important and symbolic representation in the series, representing an important theme(?).
  Already in the piece's title, \textit{Pavane pour une infante défunte}, the meaning behind is already shown.
  [Despite not literally about dead princess]
  In which ``pavane'' is a \href{https://en.wikipedia.org/wiki/Pavane}{slow dance} and the ``princess'' referring to Kaori.
  However, Kaori's health deteriorates, leading to the ``dead princess'' referred to in the title.
  

\end{enumerate}

\end{document}